% CONFIGURACIÓN GENERAL ============================================================================================
\documentclass[a4paper, 12pt]{article}

% MÁRGENES
\textheight = 21cm
\textwidth  = 16cm
\topmargin  = -0.5cm
\oddsidemargin = -0.27cm

% IDIOMA Y ENCODING
\usepackage[spanish]{babel}      % Documentos en español
\usepackage[utf8]{inputenc}     

% PAQUETES DE FORMATO Y PRESENTACIÓN
\usepackage{fancyhdr}            % Encabezados y pies
\usepackage[svgnames]{xcolor}    % Colores por nombre
\usepackage{enumerate}           % Listas personalizadas
\usepackage{multicol}            % Varias columnas
\usepackage{tocloft}             % Personalización del índice
\usepackage{etoolbox}
\usepackage{etoc}
\renewcommand{\cftsecleader}{\cftdotfill{\cftdotsep}}
\renewcommand{\cftdotsep}{1.5}
\newcommand{\HRule}{\rule{\linewidth}{0.5mm}}

% PAQUETES DE MATEMÁTICAS
\usepackage{amsmath}             

% PAQUETES PARA FIGURAS Y TABLAS
\usepackage[pdftex]{graphicx}    % Figuras
\graphicspath{{Imágenes/}}
\DeclareGraphicsExtensions{.png,.jpg,.jpeg}
\usepackage{subcaption}          % Subfiguras
\usepackage{float}               % Control H de figuras/tablas
\usepackage{booktabs}            % Tablas profesionales
\usepackage{multirow}            % Celdas múltiples en tablas
\usepackage{adjustbox}           % Escalar tablas/figuras
\usepackage[tablename=Tabla]{caption}
\captionsetup{hypcap=false}      % Evita saltos de enlace incorrectos
\usepackage{footnote}
\makesavenoteenv{tabular}        % Notas al pie dentro de tablas

% HIPERVÍNCULOS Y URLs
\usepackage[colorlinks=true, linkcolor=black, citecolor=black, urlcolor=cyan]{hyperref}
\usepackage{url}


% BIBLIOGRAFÍA Y PARCHES
\usepackage{etoolbox}
\renewcommand{\refname}{} % Quita el título “Referencias” del entorno thebibliography
\apptocmd{\sloppy}{\hbadness 10000\relax}{}{} % Evita warnings por texto muy largo
\AtBeginDocument{\let\oldref\ref\def\ref{\oldref*}}




\begin{document}

% CARÁTULA ==========================================================================================================

\begin{titlepage}
	\centering
	\includegraphics[width=0.2\textwidth]{images/Logo ITBA.png}\par
        \vspace{0.2cm}
	{\LARGE Instrumentación Biomédica II\par} 
	\vspace{0.2cm}
	{\text{Segundo Cuatrimestre 2025}\par}
    \rule{150mm}{0.1mm}
    \vspace{0.7cm}
    
    \includegraphics[width=0.45\textwidth]{images/Logo Stiffio.jpg}\par
        {\Large\bfseries Evaluación de rigidez arterial mediante la medición de velocidad de onda de pulso con fotopletismografía\par}
	\vspace{1cm}
 
    \begin{center}
	{\LARGE\bfseries Grupo 6}\par 
        \vspace{0.1cm}
        \vspace{0.2cm}
            \text{Gluj, Daniela (63404)}\par 
            \text{Jonquieres, Catalina (63369)}\par 
            \text{Orsi, Victoria (62526)}\par 

        \vspace{0.8cm}
        
	{\LARGE\bfseries Profesores}\par 
        \vspace{0.2cm}
            \text{Panza, Gustavo}\par
            \text{Igarreta, Ramón Javier}\par
            \text{Herve Birnie Scott, Santiago}\par
    \end{center}
      
	\vspace{1cm}

	{\large \textbf{Fecha de entrega:} 20/02/2026\par} 
 
\end{titlepage}



% ÍNDICE ===========================================================================================================
% Índice Principal
\newpage
\pagenumbering{gobble}
\etocsettagdepth{main}{subsubsection}   % Mostrar 'main' hasta nivel subsubsection
\etocsettagdepth{anexo}{none}           % Ocultar completamente 'anexo'
\tableofcontents                     


\vspace{0.4cm}

% Índice Anexos
\pagenumbering{gobble}
\etocsettagdepth{main}{none}         % Ocultar 'main'
\etocsettagdepth{anexo}{subsection}  % Mostrar 'anexo' hasta nivel subsection
\section*{Anexos}
\etocsettocstyle{}{} % Resetea el estilo para que no imprima el título "Índice" de nuevo automáticamente si no quieres
\tableofcontents     % Genera el segundo índice



% LISTA DE CONTRACCIONES ===========================================================================================================
\newpage
\pagenumbering{gobble}
\section*{Lista de contracciones}

\begin{description}
\item[ADC] Conversor analógico–digital.
\item[ECG] Electrocardiograma.
\item[ECV] Enfermedades cardiovasculares.
\item[HR] Frecuencia cardíaca.
\item[oPTT] Tiempo de tránsito óptico del pulso.
\item[PPG] Fotopletismografía.
\item[pPTT] Tiempo de tránsito de la onda de presión.
\item[PTT] Tiempo de tránsito del pulso.
\item[PWV] Velocidad de la onda de pulso.
\item[RA] Rigidez arterial.
\end{description}



% INTRODUCCIÓN =====================================================================================================
\newpage
\pagenumbering{arabic}
\etocdepthtag.toc{main}
\section{Introducción}

    \subsection{Enfermedades cardiovasculares y rigidez arterial}

    Las enfermedades cardiovasculares (ECV) constituyen uno de los principales problemas de salud pública a nivel mundial y representan la causa más frecuente de morbimortalidad. Según la Organización Mundial de la Salud, las ECV son responsables de aproximadamente 17,9 millones de muertes anuales, lo que equivale a cerca del 33\% del total de fallecimientos a escala global \cite{WHO_CVDs_2023}. Esta tendencia se replica en Argentina, concentrando alrededor del 27\% de las muertes registradas \cite{sac2025}. Asimismo, se estima que estas afecciones impactan en el 36,3\% de los adultos, proporción que aumenta hasta el 46,6\% en personas mayores de 45 años.

    Dada su elevada prevalencia y las severas consecuencias clínicas asociadas, la detección temprana de factores de riesgo cardiovascular resulta fundamental para reducir la carga de estas patologías. En este contexto, la rigidez arterial (RA) emerge como un marcador subclínico relevante, ya que permite identificar alteraciones vasculares antes de la aparición de manifestaciones clínicas evidentes, favoreciendo la implementación de estrategias preventivas y terapéuticas oportunas \cite{ramirez-velez2015}. 

    \vspace{0.3cm}

    La RA se define como la pérdida progresiva de la capacidad elástica de las arterias, fenómeno que ocurre de manera natural con el envejecimiento, pero que puede verse acelerado por la presencia de diversos factores de riesgo y enfermedades cardiovasculares. Este proceso refleja alteraciones estructurales y funcionales en la pared arterial, tales como el aumento del contenido de colágeno y la disminución de fibras elásticas, lo que conduce a una menor capacidad de amortiguación del pulso cardíaco. 

    Diversas patologías presentan una asociación directa con el aumento de la rigidez arterial, entre las que se incluyen la hipertensión arterial, la enfermedad coronaria, la insuficiencia cardíaca y el accidente cerebrovascular. En particular, la hipertensión arterial se destaca como uno de los principales determinantes del endurecimiento vascular, ya que la exposición crónica a niveles elevados de presión induce modificaciones estructurales y funcionales en la pared arterial \cite{gomez-sanchez2019}. De manera similar, las comorbilidades metabólicas, como la diabetes y la obesidad, se encuentran estrechamente vinculadas al aumento de la rigidez arterial. En estos pacientes, mecanismos como el estrés oxidativo, la glicación avanzada de proteínas de la matriz extracelular y la alteración de la función endotelial promueven cambios estructurales en el árbol vascular, acelerando el proceso de endurecimiento arterial. Diversos estudios poblacionales han demostrado que estas alteraciones se asocian a un mayor riesgo de eventos cardiovasculares y cerebrovasculares \cite{banegas2019, sochob2022}.



    \subsection{Velocidad de onda de pulso}
    La medición de la velocidad de la onda de pulso (PWV, por sus siglas en inglés) constituye el método no invasivo de referencia para la evaluación de la rigidez arterial \cite{aplimed, benshlomo2014}. De hecho, es considerada el \textit{gold standard} para este propósito, debido a su sólida asociación con el riesgo cardiovascular global, la incidencia de eventos ateroscleróticos y la presencia de daño de órgano blanco en etapas subclínicas \cite{hossain2025}.

    La determinación de la velocidad de onda de pulso (PWV) se basa en la medición del tiempo de tránsito del pulso (PTT) entre dos sitios arteriales, típicamente uno proximal y otro distal, como se ilustra en la Figura~\ref{fig:calculo_pwv}. Este tiempo se combina con la estimación de la distancia anatómica recorrida por la onda de presión entre ambos puntos. La PWV se calcula como el cociente entre la distancia recorrida y el tiempo de tránsito medido:

    \[\mathrm{PWV} = \frac{Dist}{PTT}\]


    \begin{figure}[H]
    \centering
    \includegraphics[width=0.6\textwidth]{images/pwv.png}
    \caption{Cálculo del tiempo de tránsito de pulso en el trayecto carótido-radial.}
    \label{fig:calculo_pwv}
    \end{figure}


    El trayecto carotídeo-femoral constituye el estándar de referencia para esta medición. En este caso, se recomienda aplicar un coeficiente de corrección de 0,8 sobre la distancia superficial medida, con el objetivo de aproximar la trayectoria real del flujo arterial. Dicho factor se encuentra respaldado por estudios que comparan mediciones externas con longitudes vasculares reales obtenidas mediante técnicas de imagen, y ha sido validado por consensos internacionales \cite{diaz2020}.

    Desde el punto de vista fisiológico, las arterias sanas presentan mayor capacidad de amortiguación de la presión pulsátil, lo que enlentece la propagación de la onda. En contraste, arterias rígidas transmiten la onda de presión a mayor velocidad, reflejándose en valores elevados de PWV, los cuales se asocian a un mayor riesgo de infarto de miocardio, accidente cerebrovascular, disecciones arteriales y aneurismas \cite{WHO_CVDs_2023, bia2019}. 
    
    Cabe destacar que los valores de PWV no son uniformes en todo el árbol arterial, sino que dependen del trayecto evaluado y de las características estructurales de los distintos segmentos vasculares. Las arterias elásticas centrales, como la aorta, presentan mayor contenido de elastina y, por lo tanto, menor velocidad de propagación en condiciones fisiológicas. En cambio, las arterias musculares periféricas poseen mayor proporción de músculo liso y menor distensibilidad, lo que se traduce en valores de PWV relativamente más elevados. Estas diferencias estructurales explican por qué la PWV varía según el segmento arterial analizado y refuerzan la importancia de estandarizar el trayecto de medición en estudios comparativos \cite{ye2016pulse}.



    \subsection{Tecnologías disponibles para la medición de la PWV}

    La rigidez arterial puede evaluarse mediante diversas técnicas no invasivas que difieren tanto en el principio físico de medición como en la forma de estimar la PWV, pudiendo realizar mediciones directas o indirectas, simultáneas o secuenciales.

    La tonometría de aplanamiento es una de las técnicas más validadas para la evaluación de la rigidez arterial y se considera el método de referencia clínico para la medición de la PWV carotídeo-femoral. Este método utiliza un sensor piezoeléctrico que se coloca sobre una arteria superficial, generalmente la carótida y la femoral, registrando de manera secuencial la forma de la onda de pulso en cada sitio. El tiempo de tránsito de la onda se calcula utilizando un electrocardiograma (ECG) como referencia temporal, y la PWV se obtiene a partir de la distancia estimada entre ambos puntos de medición. Un ejemplo ampliamente utilizado de esta tecnología es el sistema \textit{SphygmoCor}, que ofrece mediciones indirectas pero altamente reproducibles de la rigidez arterial.

    Por otro lado, la oscilometría avanzada permite evaluar la rigidez arterial a partir de registros de presión obtenidos mediante un manguito inflable, similar al utilizado en la medición convencional de la presión arterial. Sistemas como \textit{Arteriograph} utilizan un único manguito colocado en el brazo, con el cual registran las oscilaciones de presión generadas por la onda de pulso incidente y su correspondiente onda reflejada. A partir del análisis temporal entre ambos componentes de la señal, estos dispositivos estiman de manera indirecta la velocidad de la onda de pulso central basándose en modelos hemodinámicos y supuestos teóricos acerca de la propagación y reflexión de la onda de pulso en el árbol arterial \cite{aplimed}.

    Asimismo, técnicas basadas en ultrasonido Doppler también permiten la evaluación de la velocidad de la onda de pulso. En este caso, se emplean transductores ecográficos para registrar simultáneamente o de manera secuencial el flujo sanguíneo en dos segmentos arteriales, generalmente combinando mediciones en la arteria carótida y femoral o en la aorta abdominal. A partir del análisis de la señal Doppler pulsada, se determina el tiempo de tránsito entre los sitios evaluados y, junto con la estimación de la distancia anatómica correspondiente, se calcula la PWV. A diferencia de los métodos tonométricos u oscilométricos, el Doppler permite una visualización directa del vaso y del perfil de flujo, lo que puede mejorar la precisión en la identificación del punto de inicio de la onda. Sin embargo, su implementación requiere equipamiento especializado y un operador entrenado, lo que limita su aplicabilidad.



    \subsection{Comparación entre métodos basados en presión y métodos ópticos}
    Es importante distinguir entre el tiempo de tránsito de la onda de presión y el tiempo de tránsito medido mediante técnicas ópticas, ya que ambos reflejan fenómenos fisiológicos relacionados pero no equivalentes.
    
    Los sistemas basados en tonometría u oscilometría estiman el tiempo de tránsito de la onda de presión (pPTT, por sus siglas en inglés), el cual cuantifica la propagación de la energía mecánica a lo largo de la pared arterial. Este parámetro está directamente vinculado a las propiedades elásticas del vaso y a la velocidad con la que la onda de presión se transmite por el árbol arterial.
    
    En contraste, la fotopletismografía permite estimar el denominado tiempo de tránsito óptico (oPTT), el cual no mide directamente la propagación de la onda de presión, sino el cambio volumétrico local asociado a la llegada de la sangre al lecho microvascular. Para que el sensor PPG detecte dicha variación, es necesario que la sangre supere la resistencia periférica y produzca la distensión de arteriolas y capilares subcutáneos.

    Desde el punto de vista fisiológico, este proceso implica un componente viscoelástico y un tiempo de llenado tisular que introduce un retardo adicional respecto a la onda de presión registrada en el mismo sitio anatómico. Como consecuencia, la señal PPG se manifiesta sistemáticamente con un desfase temporal posterior a la onda de presión, lo que puede elongar el tiempo de tránsito medido cuando se emplean métodos ópticos \cite{ye2021evaluation}.



% JUSTIFICACIÓN DEL PROYECTO =====================================================================================================
\section{Justificación del Proyecto}
El valor de este proyecto radica en contribuir al desarrollo de herramientas complementarias para la evaluación de la rigidez arterial, abordando ciertos desafíos prácticos asociados a los dispositivos actualmente utilizados en la práctica clínica. Si bien los métodos convencionales, como la tonometría de aplanamiento, han demostrado alta validez clínica y constituyen referencias ampliamente aceptadas, su implementación requiere equipamiento costoso y capacitación técnica, lo que puede limitar su utilización en entornos ambulatorios y controles de rutina.

En este contexto, la fotopletismografía (PPG) ha emergido como una alternativa prometedora. Este método permite registrar la onda de pulso mediante principios ópticos, con potencial de integración en dispositivos portátiles, de menor costo y más fácilmente escalables \cite{karimpour2023}. Diversos estudios han demostrado que la sincronización de señales PPG obtenidas en distintos puntos corporales, así como la combinación de señales PPG y ECG, permite estimar la velocidad de la onda de pulso y otros parámetros asociados a la rigidez arterial \cite{sondeja2019, hellqvist2024}.

No obstante, la señal PPG presenta desafíos propios, tales como la sensibilidad a la presión de contacto, el adecuado posicionamiento del sensor y la interferencia por luz ambiental. Estos factores resaltan la importancia de implementar estrategias de procesamiento y corrección robustas que optimicen la precisión, estabilidad y confiabilidad de los estos sistemas.




% OBJETIVOS ========================================================================================================
\section{Objetivos}
\subsection{Objetivos de Mínima}
\begin{enumerate}
    \item \textbf{Estimación de la velocidad de onda de pulso (PWV)}\par 
    Determinar la velocidad de propagación de la onda de pulso arterial a partir de la distancia estimada entre los sensores proximal y distal, y del tiempo de tránsito del pulso obtenido mediante el análisis de las señales fotopletismográficas registradas.
    \vspace{0.1cm}
    
    \item \textbf{Determinación de la frecuencia cardíaca (HR)}\par 
    Calcular la frecuencia cardíaca mediante la detección del intervalo entre pulsos consecutivos. 
    \vspace{0.1cm}
    
    \item \textbf{Visualización de resultados en una interfaz gráfica}\par 
    Diseñar y desarrollar una aplicación de escritorio que permita visualizar los valores calculados de PWV y HR, así como graficar las señales fotopletismográficas adquiridas por los sensores proximal y distal. 
    \vspace{0.1cm}

    \item \textbf{Gestión y almacenamiento de datos del paciente}\par 
    Desarrollar e implementar una base de datos que permita almacenar y acceder a las mediciones históricas de cada paciente desde la app, facilitando el seguimiento y la comparación de resultados a lo largo del tiempo.
    \vspace{0.1cm}
    
    \item \textbf{Comparación del valor obtenido con valores fisiológicos normales}\par 
    Indicar si el valor obtenido de PWV se encuentra dentro de los rangos normales reportados en la literatura para el grupo etario del paciente.
    \vspace{0.1cm}


    \item \textbf{Evaluación estadística de las mediciones}\par
    Analizar las mediciones de HR y PWV obtenidas con el dispositivo mediante estudios estadísticos realizados sobre un conjunto de voluntarios, evaluando la estabilidad, dispersión y coherencia fisiológica de los resultados. En caso de contar con disponibilidad de un dispositivo clínico de referencia, se realizará adicionalmente un análisis comparativo para evaluar la concordancia de las mediciones obtenidas.
    \vspace{0.1cm}


    \item \textbf{Diseño y desarrollo de la carcasa del dispositivo}\par 
    Diseñar y fabricar una carcasa que asegure la correcta fijación de los componentes internos. El diseño debe contemplar un compartimiento de baterías aislado y de fácil acceso para el usuario, así como la integración de una pantalla táctil que posibilite el ingreso de los datos del paciente sin necesidad de la app.
    \vspace{0.1cm}

    \item \textbf{Encapsulado de sensores}\par 
    Diseñar y desarrollar un encapsulado robusto para los sensores proximal y distal que asegure su correcto funcionamiento y mejore su durabilidad en un contexto clínico, evitando el contacto directo de la electrónica con la piel del paciente y garantizando la comodidad durante su uso.
    \vspace{0.1cm}
    
\end{enumerate}


\subsection{Objetivos de Máxima}
\begin{enumerate}
    \item \textbf{Autonomía del dispositivo}\par 
    Calcular y visualizar los valores de HR, PWV y las curvas fotopletismográficas directamente desde el dispositivo, sin necesidad de utilizar una app externa ni conexión a WiFi, permitiendo su uso de forma autónoma.
    \vspace{0.1cm}

    \item \textbf{Generación de reportes de medición}\par 
    Incorporar en la app la opción de generar reportes de medición en formato PDF, destinados tanto a su impresión como a su envío digital al paciente, incluyendo los datos identificatorios del mismo y los valores obtenidos durante el estudio.
    \vspace{0.1cm}

    \item \textbf{Búsqueda de registros}\par 
    Facilitar la consulta de las mediciones almacenadas en la base de datos mediante la implementación de un sistema de búsqueda que permita localizar registros por nombre, apellido o DNI, incorporando además filtros por rango de fechas para una gestión eficiente del historial clínico.
    \vspace{0.1cm}


\end{enumerate}

    

% DESCRIPCIÓN GENERAL ==============================================================================================
\section{Descripción General y Uso Previsto}
El presente proyecto se centra en el desarrollo de un dispositivo biomédico de bajo costo diseñado para la estimación no invasiva de la rigidez arterial, utilizando como indicador principal la velocidad de onda de pulso. 

El sistema emplea dos sensores ópticos infrarrojos para capturar la onda de pulso en dos puntos estratégicos del trayecto arterial. El sensor proximal se coloca a nivel de la arteria carótida en el cuello, mientras que el sensor distal va colocado sobre de la arteria radial en la muñeca. La selección de estos puntos anatómicos responde a criterios de ergonomía y calidad de señal. Ambas son zonas de fácil acceso para el operador, ofrecen comodidad al paciente y presentan una baja densidad capilar, lo que optimiza la lectura óptica.

El principio de operación del dispositivo consiste en calcular el PTT a partir del desfase temporal entre las señales proximal y distal. Combinando este valor con la distancia anatómica entre los sensores, el sistema computa la PWV. La distancia utilizada para el cálculo se estima a partir de la altura del paciente. Adicionalmente, a partir de la señal del sensor distal, se obtiene la HR como variable adicional.

\vspace{0.2cm}

El diagrama de bloques del funcionamiento general del sistema se ilustra en la Figura~\ref{fig:Diagrama de Bloques}.

\begin{figure}[H]
\centering
\includegraphics[width=1\textwidth]{images/Diagrama de Bloques.png}
\caption{Diagrama de bloque general del funcionamiento  del dispositivo.}
\label{fig:Diagrama de Bloques}
\end{figure}

\vspace{0.2cm}

Para adaptarse a distintos flujos de trabajo, el dispositivo ofrece dos modos de uso:
\begin{enumerate}
    \item \textbf{Modo Estudio Clínico (Vinculado):} El dispositivo establece comunicación con \textit{StiffioApp}, una aplicación de escritorio que permite no solo la visualización en tiempo real de las señales, sino también el almacenamiento de los registros para el seguimiento y análisis de la evolución clínica del paciente.
    \item \textbf{Modo Test Rápida (Autónomo):} Permite operar el equipo de manera independiente a través de su pantalla táctil integrada, sin requerir conexión a una PC. En este modo, sólo se ingresan los datos del paciente que son estrictamente necesarios para el cálculo, por lo que es ideal para mediciones rápidas. 
\end{enumerate}

El dispositivo está dirigido al uso en consultorio clínico. La propuesta no busca reemplazar a los sistemas de referencia hospitalarios, sino brindar una herramienta de apoyo confiable, pensada para médicos clínicos que deseen incorporar la evaluación de rigidez arterial en controles de rutina. 






% MATERIALES Y MÉTODOS =============================================================================================
\section{Materiales y Métodos}

    % HARDWARE -----------------------------------------------------------------------------------------------------
    \subsection{Hardware}
        \subsubsection{Circuito Electrónico}
        El  sistema de procesamiento esta constituido por un microcontrolador ESP32 (modelo DEVKIT-V1), seleccionado por su arquitectura de doble núcleo y elevada precisión temporal, característica imprescindible para medir con exactitud los intervalos entre señales fotopletismográficas y, de este modo, calcular correctamente el tiempo de tránsito del pulso. Además, este microcontrolador ofrece múltiples canales de comunicación I²C, lo que facilita la conexión simultánea de los periféricos digitales utilizados en el dispositivo.

        Para la adquisición de las señales fotopletismográficas se emplearon dos sensores MAX30102, ubicados en dos posiciones anatómicas diferentes (proximal y distal). Ambos módulos se comunican con el ESP32 mediante el protocolo I²C. Para evitar conflictos de dirección y garantizar la lectura simultánea, se implementaron dos buses independientes, asignando pares de pines distintos para las líneas de datos (SDA) y reloj (SCL) de cada sensor. Adicionalmente, se integraron etapas de filtrado en cada puerto de conexión, compuestas por un arreglo de capacitores en paralelo (10 $\mu$F y 100 nF). Estos actúan como condensadores de desacople, filtrando el ruido de alta frecuencia y estabilizando la tensión de suministro ante variaciones rápidas en el consumo de los sensores.

        Por otro lado, se integró una  pantalla TFT táctil de 4 pulgadas con comunicación SPI. Este módulo de visualización permite graficar las ondas de pulso con alta resolución y recibir comandos del usuario, otorgando autonomía al dispositivo. De manera análoga a los sensores, se incorporó un capacitor de 100 nF en la entrada de alimentación de la pantalla para filtrar ruido de alta frecuencia y evitar caídas de tensión transitorias que podrían manifestarse como parpadeos o variaciones indeseadas en el brillo del display.

        Adicionalmente, se incorporó un LED rojo conectado a la línea de alimentación con el objetivo de indicar el estado de encendido del equipo. Asimismo, a modo de retroalimentación auditiva ante la presión de los botones de la pantalla y para la emisión de alertas sonoras, se incluyó un buzzer activo. Dado que la corriente requerida por el buzzer podría superar la capacidad de entrega segura de los pines del microcontrolador, se implementó un circuito driver utilizando un transistor BC548 en configuración de conmutador. Dicho transistor es controlado por un pin digital del ESP32 a través de una resistencia de base de 1 k$\Omega$, actuando como un interruptor electrónico que permite manejar la carga del buzzer sin comprometer la integridad del microcontrolador.

        La alimentación del sistema se realiza mediante 4 baterías AA, conectadas a la placa principal mediante un conector Molex de 2 pines. La tensión de entrada es luego acondicionada por un regulador de voltaje que entrega 3.3 V estables a todos los componentes. Al igual que en los sensores, se incluyeron dos capacitores de desacople (10uF y 100nF) a la salida del regulador para amortiguar transitorios de tensión.

        \vspace{0.2cm}

        En la Figura \ref{fig:Circuito} se presenta el esquema completo del circuito diseñado. La asignación de pines utilizada para la conexión de los distintos componentes se detalla en el Anexo \ref{anexo:ConexionPines}.

        \begin{figure}[H]
        \centering
        \includegraphics[width=1\textwidth]{images/Circuito.jpeg}
        \captionsetup{margin=1.5cm, justification=centering}
        \caption{Esquema del circuito implementado (las entradas de tensión se corresponden con la salida del regulador).}
        \label{fig:Circuito}
        \end{figure}

        \vspace{0.2cm}

        Finalmente, se confeccionó el circuito descripto sobre una placa de cobre doble faz. La elección de una PCB de dos capas permitió optimizar la distribución del cableado y ubicar estratégicamente todos los componentes mencionados de acuerdo con el diseño de la carcasa. El resultado de la placa final se muestra en la Figura \ref{fig:Placa}.

        
        \begin{figure}[H]
        \centering
        
        \begin{minipage}[b]{0.45\textwidth}
            \centering
            \includegraphics[width=\textwidth]{images/Placa Front.png}
        \end{minipage}
        \hspace{0.01cm} 
        \begin{minipage}[b]{0.45\textwidth}
            \centering
            \includegraphics[width=\textwidth]{images/Placa Back.png}
        \end{minipage}
        
        \captionsetup{margin=1.5cm, justification=centering}
        \caption{Frente (izquierda) y contrafrente (derecha) de la PCB.}
        \label{fig:Placa}
        \end{figure}

    

        
        \subsubsection{Diseño y Fabricación de Carcasa}
        El diseño de la carcasa del dispositivo se realizó utilizando el software de diseño asistido por computadora \textit{SolidWorks}. El modelo tridimensional completo, así como las vistas que detallan la geometría y las características constructivas de la pieza, se presentan en el Anexo \ref{anexo:ModeloCarcasa}.
        
        \vspace{0.2cm}

        El diseño fue concebido con el objetivo de alojar de manera segura los componentes electrónicos y, al mismo tiempo, facilitar la interacción del usuario con el dispositivo. En uno de los laterales de la carcasa se incorporó un orificio destinado al interruptor de encendido y apagado, mientras que en el panel trasero se dispusieron dos orificios para colocar los conectores de los sensores.

        La tapa superior cuenta con una abertura rectangular que permite la correcta visualización de la pantalla, un orificio adyacente para el LED indicador de encendido y un segundo orificio de menor tamaño para permitir la salida del sonido generado por el buzzer. Además, incorpora dos soportes elásticos destinados a sujetar el lápiz, el cual facilita la interacción del usuario con la interfaz táctil.
        La tapa se fija al cuerpo de la carcasa mediante tornillos con cabeza Allen, los cuales atraviesan también la placa electrónica, asegurando su correcta sujeción. Esta configuración dificulta el acceso del usuario a los componentes internos, limitando la apertura del dispositivo únicamente a situaciones que requieran mantenimiento o servicio técnico.

        Por otro lado, en la parte inferior de la carcasa se incorporó un compartimiento independiente, aislado del área donde se encuentra la placa electrónica, destinado al alojamiento de las baterías. Este compartimiento incluye un portapilas para las cuatro baterías AA y presenta un diseño de fácil acceso mediante una tapa deslizante. El cierre de la misma se asegura a través de un mecanismo de encastre a presión. 

        Finalmente, para garantizar la fijación del dispositivo a la mesa de trabajo y evitar su desplazamiento, se dispusieron una serie de ventosas en la base de la carcasa. Este sistema de anclaje previene caídas accidentales provocadas por la posible tensión al manipular los cables de los sensores durante las mediciones.

        \vspace{0.2cm}

        La totalidad de la carcasa se fabricó mediante impresión 3D utilizando filamento PLA negro. Las dimensión final de la misma es de 14x11x5.5 cm. En la Figura \ref{fig:Carcasa} se exhibe el dispositivo final ensamblado, así como el detalle de sus partes anteriormente descriptas.

        \begin{figure}[H]
        \centering
        
        \begin{minipage}[b]{0.3\textwidth}
            \centering
            \includegraphics[width=\textwidth]{images/Caja 1.png}
        \end{minipage}
        \hspace{0.01cm} 
        \begin{minipage}[b]{0.3\textwidth}
            \centering
            \includegraphics[width=\textwidth]{images/Caja 2.png}
        \end{minipage}
        \hspace{0.01cm} 
        \begin{minipage}[b]{0.3\textwidth}
            \centering
            \includegraphics[width=\textwidth]{images/Bateria.png}
        \end{minipage}
        
        \captionsetup{margin=2cm, justification=centering}
        \caption{Carcasa impresa en 3D: vista frontal (izquierda), posterior (centro) e inferior (derecha).}
        \label{fig:Carcasa}
        \end{figure}

        
        Para la conexión de los sensores al dispositivo, se eligieron conectores metálicos circulares de 4 pines. Estos incluyen un mecanismo de acople roscado que garantiza una sujeción firme y previene desconexiones accidentales durante el uso. 
        
        Adicionalmente, con el fin de guiar al usuario en su correcta conexión, se implementó un sistema de identificación por colores. Se dispuso un indicador visual sobre cada puerto en correspondencia con el sensor asignado: rojo para el proximal y rosa para el distal, tal como se ilustra en la Figura \ref{fig:Colores Conectores}.

        \begin{figure}[H]
        \centering
        \includegraphics[width=0.7\textwidth]{images/Colores Puertos.png}
        \captionsetup{margin=1.5cm, justification=centering}
        \caption{Código de colores para la correcta conexión de los sensores al dispositivo.}
        \label{fig:Colores Conectores}
        \end{figure}

        \vspace{0.2cm}
        

        \subsubsection{Encapsulado de Sensores}
        Para el encapsulado de los sensores MAX30102 se diseñó un conjunto compuesto por dos piezas complementarias, orientadas a garantizar tanto la correcta funcionalidad óptica como la ergonomía del dispositivo.

        La pieza frontal corresponde a una carcasa rígida fabricada mediante impresión 3D en PLA, cuyo objetivo es aislar el componente electrónico del contacto directo con la piel. Esta pieza presenta una ventana óptica de dimensiones acotadas, diseñada para asegurar la adecuada transmisión de luz hacia el tejido y la correcta recepción de la señal reflejada. 
        Además, con el objetivo de proteger la integridad del cableado, se incorporó en la base de la pieza un prensacables, el cual actúa como alivio de tensión mecánica, evitando la rotación axial del cable y reduciendo el riesgo de fatiga o ruptura de las soldaduras internas por tracción.
        Detalles adicionales sobre el diseño y modelado de esta pieza pueden observarse en el Anexo \ref{anexo:ModeloSensor}.

        La pieza trasera consiste en una funda de silicona flexible en la cual se inserta la pieza frontal descripta anteriormente. Esta funda cumple la función de ocultar las conexiones eléctricas, otorgando al conjunto una estética más limpia y segura. Asimismo, esta estructura incorpora en sus laterales dos elásticos con cierres de velcro que permiten un ajuste firme alrededor del cuello o la muñeca del paciente, adaptándose a la medida que sea necesaria.

        \vspace{0.4cm}

        Ambas piezas que conforman el encapsulado pueden observarse en detalle en las Figuras \ref{fig:Sensor Front} y \ref{fig:Sensor Back}, mientras que el ensamble final del sensor se presenta en la Figura \ref{fig:Sensor}.
    

        \begin{figure}[H]
        \centering
        
        \begin{minipage}[b]{0.35\textwidth}
            \centering
            \includegraphics[width=\textwidth]{images/Sensor Front.png}
        \end{minipage}
        \hspace{0.01cm} 
        \begin{minipage}[b]{0.35\textwidth}
            \centering
            \includegraphics[width=\textwidth]{images/Sensor Front 2.png}
        \end{minipage}
        
        \captionsetup{margin=1.5cm, justification=centering}
        \caption{Vista frontal (izquierda) y posterior (derecha) de la pieza frontal para el encapsulado del sensor.}
        \label{fig:Sensor Front}
        \end{figure}



        \begin{figure}[H]
        \centering
        
        \begin{minipage}[b]{0.45\textwidth}
            \centering
            \includegraphics[width=\textwidth]{images/Sensor Back.png}
        \end{minipage}
        \hspace{0.01cm} 
        \begin{minipage}[b]{0.45\textwidth}
            \centering
            \includegraphics[width=\textwidth]{images/Sensor Back 2.png}
        \end{minipage}
        
        \captionsetup{margin=1.5cm, justification=centering}
        \caption{Vista frontal (izquierda) y posterior (derecha) de la pieza frontal para el encapsulado del sensor.}
        \label{fig:Sensor Back}
        \end{figure}



        \begin{figure}[H]
        \centering
        \includegraphics[width=0.5\textwidth]{images/Sensor.png}
        \captionsetup{margin=1.5cm, justification=centering}
        \caption{Encapsulado completo y ensamblado del sensor MAX30102}
        \label{fig:Sensor}
        \end{figure}

        \vspace{0.2cm}

        
    % SOFTWARE -----------------------------------------------------------------------------------------------------
    \subsection{Software}
        \subsubsection{Arquitectura del sistema y flujo de datos}
        El firmware desarrollado para el microcontrolador ESP32 aprovecha su arquitectura de doble núcleo para ejecutar en paralelo la adquisición de señales y la gestión de la interfaz gráfica. En el Core 0 se implementa la tarea de muestreo de los sensores MAX30102, ubicados en posiciones proximal y distal, a una frecuencia de 50 Hz (intervalo de 20 ms). Cada muestra es registrada con un timestamp absoluto, lo que permite calcular con precisión el PTT y PWV.

        El sistema opera en dos modos diferenciados: \textbf{Test Rápido} (autónomo) y \textbf{Estudio Clínico} (vinculado a la aplicación de escritorio). En el modo Test Rápido, todo el procesamiento y la visualización se realizan íntegramente en el dispositivo, permitiendo el ingreso de datos básicos y la visualización de resultados en la pantalla táctil. En el modo Estudio Clínico, el dispositivo se conecta por WiFi a la aplicación de escritorio mediante WebSockets, enviando los datos en tiempo real en formato JSON para su almacenamiento y análisis avanzado.

        La interfaz gráfica embebida se estructura como una máquina de estados, permitiendo la navegación secuencial entre pantallas, el ingreso de datos y la visualización de métricas y curvas. El sistema incluye alertas visuales y sonoras ante errores, desconexión de sensores o valores fuera de rango, garantizando robustez y facilidad de uso.


        En ambos modos, el sistema monitorea en tiempo real el estado de conexión de los sensores y la validez de los resultados, generando alertas y permitiendo al usuario reiniciar la medición de manera sencilla. La arquitectura modular del software facilita la adaptación a distintos flujos de trabajo y asegura la sincronización precisa de las señales adquiridas. 

        \vspace{0.3cm}

        El diagrama de bloques de la arquitectura general del sistema se muestra en la Figura \ref{fig:diagrama_software_menu}.

        \vspace{0.2cm}

        \begin{figure}[H]
        \centering
        \includegraphics[width=1\textwidth]{images/DiagramaSoftwareMenu.png}
        \captionsetup{margin=1.5cm, justification=centering}
        \caption{Diagrama de bloques del software del sistema, mostrando los dos modos de operación: Test Rápido y Estudio Clínico, y el flujo de datos entre los principales módulos funcionales.}
        \label{fig:diagrama_software_menu}
        \end{figure}

        \subsubsection{Filtrado digital de las señales}
        Para mejorar la calidad de la señal fotopletismográfica adquirida por los sensores MAX30102, se implementó un preprocesamiento digital en el microcontrolador. La señal original proviene del fotodiodo integrado en el sensor, cuya lectura es digitalizada mediante un conversor analógico–digital (ADC) interno de 18 bits. Este valor corresponde a la luz infrarroja reflejada por el tejido e incluye un componente DC de gran magnitud asociado a la absorción constante de luz, sobre el cual se superpone una componente AC de muy baja amplitud generada por las variaciones pulsátiles del volumen sanguíneo.

        \vspace{0.1cm}

        En la Figura~\ref{fig:PPG_reflectance} se observa la diferencia entre los modos de adquisición por transmisión (A) y por reflectancia (B). En el modo de transmisión, el pie de onda fisiológico corresponde a un mínimo local de la señal. Sin embargo, en el modo reflectivo utilizado en este dispositivo, el pie de onda fisiológico se manifiesta como un máximo local (punto e en la imagen). Para unificar el procesamiento y facilitar la detección automática, la señal es invertida digitalmente antes de la detección de eventos, de modo que el pie de onda pueda ser tratado como un mínimo local, coherente con la mayoría de los algoritmos descritos en la literatura.

        \begin{figure}[H]
        \centering
        \includegraphics[width=0.9\textwidth]{images/PPG_reflectance_vs_transmission.jpg}
        \captionsetup{margin=1.5cm, justification=centering}
        \caption{Modos de adquisición de señal PPG: transmisión (A) y reflexión (B). En el modo reflectivo, el pie de onda fisiológico (e) se manifiesta como un máximo local. Imagen adaptada de \cite{paperPPG}.}
        \label{fig:PPG_reflectance}
        \end{figure}

        Antes del filtrado, se realiza una normalización dividiendo los valores crudos por el máximo representable en 18 bits, permitiendo así escalar la señal a un rango uniforme y adecuado para su graficación y procesamiento.

        Una vez normalizada, la señal se somete a un filtrado secuencial destinado a eliminar las componentes no fisiológicas. En primer lugar, se aplica un filtro pasa altos de primer orden con una frecuencia de corte de 0,5 Hz, cuyo objetivo es suprimir el componente DC y variaciones lentas asociadas al movimiento o cambios de perfusión. A continuación, la salida del pasa altos se procesa mediante un filtro pasa bajos de primer orden con una frecuencia de corte de 5 Hz, que atenúa el ruido de alta frecuencia y preserva únicamente el contenido característico de la onda pulsátil. Finalmente, se implementa una media móvil de cuatro muestras para un suavizado adicional más robusto, mejorando la estabilidad de la señal y la detección de eventos.

        En conjunto, este filtrado pasa banda resulta adecuado para el rango fisiológico de la señal PPG y mejora significativamente la calidad para la detección de los pulsos y el cálculo de los parámetros derivados.


        \subsubsection{Cálculo de frecuencia cardíaca}
        La señal filtrada ingresa al algoritmo de detección de latidos, que se basa en la identificación de pies de onda en la señal distal (sensor ubicado en la muñeca). Para ello, se detectan mínimos locales en la señal invertida, correspondientes al inicio del ascenso sistólico (punto e en la Figura~\ref{fig:PPG_reflectance}). Para que un pico sea detectado se exige que el valor actual sea menor que sus vecinos inmediatos y que se encuentre por debajo de un umbral adaptativo calculado dinámicamente sobre una ventana móvil. Además, se verifica que la derivada cambie de signo (de negativa a positiva) para confirmar la presencia de un pie de onda fisiológico y evitar la detección de artefactos.

        Cada vez que se detecta un pie de onda válido, se calcula el intervalo RR en milisegundos como la diferencia temporal entre dos eventos consecutivos. Estos valores se almacenan en un buffer FIFO circular de 15 muestras, que se actualiza en cada nuevo latido detectado. Para obtener una estimación robusta y estable de la frecuencia cardíaca, se calcula la mediana de los valores almacenados en el buffer. Finalmente el valor de frecuencia cardíaca mostrado en pantalla se calcula con la siguiente relación:
        
        \[\mathrm{HR} = \frac{60000}{RR_{\text{mediana}}}\]

        \vspace{0.3cm}
        
        donde $RR_{\text{mediana}}$ es el valor mediano de los intervalos RR en milisegundos. 

        \vspace{0.3cm}

        El sistema exige acumular un mínimo de latidos válidos antes de mostrar el resultado, lo que garantiza la estabilidad de la medición y evita fluctuaciones erráticas. Si los sensores pierden contacto o la señal cae por debajo del umbral de apoyo, todos los promedios y contadores se reinician, obligando a reconstruir la secuencia de latidos antes de exponer un nuevo valor. Este mecanismo asegura que la frecuencia cardíaca mostrada represente una estimación suave y robusta, permitiendo además reiniciar automáticamente la medición al cambiar de paciente.

        
        \subsubsection{Cálculo de velocidad de onda de pulso}
        El cálculo de la velocidad de onda de pulso (PWV) se realiza emparejando cada pie de onda detectado en la señal proximal con el siguiente pie de la señal distal, obteniendo así el tiempo de tránsito del pulso ($\Delta t$) como la diferencia temporal entre ambos eventos. Los valores de $\Delta t$ obtenidos se almacenan secuencialmente en un buffer no circular de 25 muestras. Una vez completado el buffer, se calcula la mediana de estos 25 valores, que es utilizada para estimar y mostrar el valor de PWV. Este enfoque permite obtener una estimación más robusta y estable, minimizando el impacto de valores atípicos o artefactos.

        La velocidad de onda de pulso se calcula mediante la siguiente fórmula, que relaciona la distancia anatómica recorrida por la onda ($d$) y el tiempo de tránsito medido:

        \[
        \mathrm{PWV} = \frac{d}{\Delta t}
        \]

        \vspace{0.1cm}

        Antes de actualizar la PWV calculada, el algoritmo exige acumular veinticinco valores válidos de $\Delta t$, asegurando que la estimación presentada en la interfaz sea estable y no esté afectada por fluctuaciones instantáneas. El sistema monitorea en todo momento el estado de conexión de los sensores y la validez de los resultados, generando alertas visuales y sonoras ante cualquier error o desconexión, y permitiendo al usuario reiniciar la medición de manera sencilla.


        \paragraph{Estimación de la distancia}\mbox{}\\

        La estimación de la distancia anatómica recorrida por la onda de pulso se realiza multiplicando la altura del usuario por un factor fijo de 0,436. Este coeficiente proviene de mediciones anatómicas realizadas sobre 54 sujetos de distintas edades, donde se midió con una cinta métrica flexible la distancia externa entre el punto palpable de la arteria carótida en el cuello y el punto palpable de la arteria radial en la muñeca, como se muestra en la Figura~\ref{fig:Arterias}. Más detalles sobre el procedimiento de medición se presentan en el Anexo~\ref{anexo:FactorCorreccion}.
        
        Se adoptó esta geometría carótidorradial porque el sensor óptico no resultó práctico en la región púbica, donde permitiria una medición mas estandar carótido-femoral. El vello y la ergonomía dificultan la correcta alineación en el trayecto, comprometiendo la comodidad del paciente. Además, la literatura evidencia que la PWV carótida–radial es reproducible y presenta correlación con marcadores fisiopatológicos centrales, admitiendo consideraciones específicas para su interpretación frente a la central, pero permitiendo una alternativa fiable y robusta en dispositivos ópticos portátiles.\cite{nitta2024}\cite{wang2023}
        

        \begin{figure}[H]
        \centering
        \includegraphics[width=0.4\textwidth]{images/Arterias.jpg}
        \captionsetup{margin=1.5cm, justification=centering}
        \caption{Distancia recorrida por la onda de pulso en A. Aorta ascendente y Tronco braquilocefálico  B. Arteria Carótida; C. Arteria braquial y arteria radial}
        \label{fig:Arterias}
        \end{figure}
        


    % INTERFACES DE USUARIO -----------------------------------------------------------------------------------------------------
    \subsection{Interfaces de usuario}

        \subsubsection{Interfaz táctil en display integrado (Modo Test Rápido)}
        Con el objetivo de permitir el uso autónomo del dispositivo, se desarrolló una interfaz gráfica embebida en el display táctil integrado. Esta interfaz posibilita la utilización de todas las funcionalidades principales del sistema sin necesidad de conexión a una PC, a excepción del almacenamiento y la gestión avanzada de datos, que se realizan exclusivamente mediante la aplicación de escritorio \textit{StiffioApp}.

        La interfaz fue desarrollada en lenguaje C++ utilizando el entorno Arduino IDE. Para el control gráfico del display se empleó la librería \texttt{TFT\_eSPI.h}, encargada del renderizado de texto, botones, gráficos y actualización dinámica de pantalla, mientras que para la detección y procesamiento de las coordenadas del panel táctil resistivo se utilizó la librería \texttt{XPT2046\_Touchscreen.h}.  

        Al encender el dispositivo, se presenta un menú inicial en el cual el usuario debe seleccionar el modo de funcionamiento: ESTUDIO CLÍNICO o TEST RÁPIDO, tal como se muestra en la Figura~\ref{fig:MenuDisplay}.


        \begin{figure}[H]
        \centering
        \includegraphics[width=0.7\textwidth]{images/Menu Display.png}
        \captionsetup{margin=1.5cm, justification=centering}
        \caption{Menú principal de la interfaz táctil en el display integrado. Selección de modo de uso.}
        \label{fig:MenuDisplay}
        \end{figure}


        En caso de seleccionar el modo \textbf{ESTUDIO CLÍNICO}, el dispositivo establece conexión WiFi y muestra un mensaje indicando que debe abrirse la aplicación de escritorio \textit{StiffioApp}. En este modo, el ingreso de datos del paciente, la visualización detallada de señales y el almacenamiento de resultados se gestionan exclusivamente desde la aplicación externa. El modo ESTUDIO CLÍNICO solo puede ser utilizado desde la PC por medio de la aplicación.

        Si el usuario selecciona el modo \textbf{TEST RÁPIDO}, la interfaz en la pantalla solicita la carga de los datos estrictamente necesarios para el cálculo, que son edad y altura. El ingreso de los mismos se realiza mediante un teclado numérico táctil como el que se ilustra en la Figura~\ref{fig:TecladoNumerico}. El dato de entrada es verificado antes de permitir avanzar, y en caso de ser inválido se muestra una alerta. Una vez completados los campos requeridos, el usuario debe presionar el botón de continuar (flecha verde) para iniciar la medición. 

        
        \begin{figure}[H]
        \centering
        \includegraphics[width=0.7\textwidth]{images/Teclado Numerico.png}
        \captionsetup{margin=1.5cm, justification=centering}
        \caption{Teclado numérico para el ingreso de datos de la interfaz táctil en el display integrado.}
        \label{fig:TecladoNumerico}
        \end{figure}


        En la Figura  \ref{fig:MedicionDisplay} se puede observar la ventana principal de medición, donde se muestran los resultados del cálculo. El usuario puede seleccionar el modo de visualización deseado presionando los íconos ubicados en la esquina superior derecha del display:
        \begin{itemize}
            \item \textbf{Modo Métricas}: Muestra sólo los valores numéricos calculados (HR y PWV).
            \item \textbf{Modo Curvas}: Además de las métricas, muestra en tiempo real las señales pletismográficas provenientes de ambos sensores.
        \end{itemize}

        A modo indicativo, los valores de PWV se muestran en color rojo cuando se encuentran fuera del rango considerado normal para el grupo etario del paciente, y en color verde cuando se encuentran dentro de los valores de referencia.


        \begin{figure}[H]
        \centering
        \includegraphics[width=0.7\textwidth]{images/Resultados Display.png}
        \captionsetup{margin=1.5cm, justification=centering}
        \caption{Pantalla de visualización de resultados de la interfaz táctil en el display integrado.}
        \label{fig:MedicionDisplay}
        \end{figure}
    

        Para finalizar la medición y regresar al menú principal, el usuario debe presionar el botón SALIR, ubicado en la esquina inferior derecha de la interfaz.
    


        \subsubsection{Aplicación de escritorio (Modo Estudio Clínico)}
        Además de la interfaz embebida en el dispositivo, se diseñó una aplicación de escritorio denominada \textit{StiffioApp}, destinada a la visualización, almacenamiento y gestión de datos clínicos. Esta aplicación fue desarrollada en Python utilizando el framework PyQt.

        \vspace{0.2cm}


        Para utilizar esta aplicación, el usuario debe seleccionar la opción ESTUDIO CLÍNICO desde el display del dispositivo, tal como se mostró previamente en la Figura~\ref{fig:MenuDisplay}. Al hacerlo, el sistema establece comunicación vía WiFi con la computadora para la transmisión de datos.

        Al iniciar la aplicación, se presenta la pantalla mostrada en la Figura~\ref{fig:MenuInterfaz}, donde se ofrecen dos opciones. La opción Comenzar permite iniciar una nueva medición utilizando el dispositivo, mientras que Historial de Mediciones posibilita acceder a los registros previamente almacenados.


        \begin{figure}[H]
        \centering
        \includegraphics[width=0.9\textwidth]{images/Menu Interfaz.png}
        \captionsetup{margin=1.5cm, justification=centering}
        \caption{Menú principal de \textit{StiffioApp}.}
        \label{fig:MenuInterfaz}
        \end{figure}


        Al seleccionar \textbf{Comenzar}, se despliega una ventana destinada a la carga de datos del paciente, donde es posible ingresar nombre, apellido, DNI, edad, altura y sexo antes de comenzar la adquisición. Asimismo, se dispone de un campo adicional para incorporar observaciones clínicas relevantes, tales como síntomas o antecedentes médicos, en caso de que el usuario lo considere pertinente. El diseño y la disposición de los elementos de esta ventana se muestran en la Figura~\ref{fig:DatosInterfaz}.



        \begin{figure}[H]
        \centering
        \includegraphics[width=0.9\textwidth]{images/Interfaz Datos.png}
        \caption{Pantalla de carga de datos de \textit{StiffioApp}.}
        \label{fig:DatosInterfaz}
        \end{figure}


        Una vez completados los datos y confirmada la acción, se accede a la ventana principal de la aplicación, mostrada en la Figura~\ref{fig:MedicionInterfaz}. En esta interfaz se concentran las funcionalidades operativas del sistema.
        
        El botón de control Iniciar Medición activa la adquisición y visualización en tiempo real de las señales en el panel derecho. Cada señal se representa con el mismo color que la funda de silicona del sensor correspondiente, facilitando su identificación visual. Debajo de las gráficas se muestran los valores calculados de frecuencia cardíaca (HR) y velocidad de onda de pulso (PWV). Al presionar  el botón Detener Medición, la adquisición se detiene y tanto las señales como los valores calculados permanecen congelados en pantalla para su análisis.
        
        En el panel lateral izquierdo, debajo de los datos del paciente, se incluye un gráfico de PWV en función de la edad. Sobre dicho gráfico se marca con un punto el valor de PWV medido, permitiendo visualizar si se encuentra dentro de los rangos considerados normales para el grupo etario correspondiente (zona verde) o fuera de ellos (zona roja).

        Los parámetros medidos pueden almacenarse junto con los datos del paciente mediante el botón Guardar Medición. De este modo, se genera un registro histórico que permite conservar las mediciones en el tiempo, funcionando como una historia clínica digital.
        
        Si se desea realizar una nueva medición, el botón Nuevo Paciente redirige nuevamente a la ventana de carga de datos. Para regresar al menú principal, se dispone de un botón SALIR ubicado justo al lado.


        \begin{figure}[H]
        \centering
        \includegraphics[width=0.9\textwidth]{images/Interfaz Main.jpeg}
        \caption{Pantalla de visualización de resultados de \textit{StiffioApp}}
        \label{fig:MedicionInterfaz}
        \end{figure}

        \vspace{0.3cm}

        Por otro lado, al seleccionar la opción \textbf{Historial de Mediciones} en el menú principal, se accede a la pantalla mostrada en la Figura~\ref{fig:HistorialInterfaz}. En esta sección es posible visualizar la lista completa de mediciones previamente almacenadas en la base de datos del sistema. 
        
        La interfaz presenta los registros en formato tabular, incluyendo los datos identificatorios del paciente y los principales parámetros obtenidos durante cada estudio. Para localizar un registro específico, el usuario dispone de herramientas de búsqueda y filtrado que permiten acotar los resultados según distintos criterios. Asimismo, puede desplazarse verticalmente mediante la barra de desplazamiento para navegar entre todos los registros disponibles.
        
        Si se desea exportar una medición, se debe presionar el ícono de descarga correspondiente al registro seleccionado. Esta acción genera automáticamente un reporte en formato PDF que incluye los datos del paciente y los resultados obtenidos durante la medición. Se puede observar un ejemplo de dicho reporte en el Anexo \ref{anexo:ReportePDF}. 
        
        En caso de que se requiera eliminar un registro, se puede presionar el ícono de borrado asociado. Antes de proceder, la aplicación solicita una confirmación para evitar perdida de información accidental.
        
        Finalmente, para salir del historial y regresar al menú principal, el usuario debe presionar el botón Volver, ubicado en la esquina inferior izquierda de la pantalla.


        \begin{figure}[H]
        \centering
        \includegraphics[width=0.9\textwidth]{images/Historial Interfaz.jpeg}
        \caption{Pantalla de historial de mediciones de \textit{StiffioApp}}
        \label{fig:HistorialInterfaz}
        \end{figure}
        
        
        



    
    
    \subsection{Metodología de Evaluación y Validación}
    En esta sección se describe el enfoque metodológico adoptado para la evaluación del dispositivo, detallando las condiciones bajo las cuales se realizaron las mediciones y los criterios utilizados para su análisis.

        \subsubsection{Protocolo de Medición}

        \begin{enumerate}

        \item \textbf{Preparación del paciente:}
        \begin{itemize}
            \item Explicar el procedimiento al paciente y obtener el consentimiento informado correspondiente.
            \item Colocar al paciente en decúbito supino, con los miembros superiores relajados, en un ambiente tranquilo y con temperatura controlada (22–26\,°C).
            \item Indicar evitar la ingesta de cafeína, nicotina, alcohol o la realización de ejercicio intenso al menos 3 horas antes de la medición.
            \item Permitir un período de reposo de 5 a 10 minutos previo al inicio, con el fin de alcanzar estabilidad hemodinámica y respiración normal.
        \end{itemize}

        \item \textbf{Preparación del equipo:}
        \begin{itemize}
            \item Encender el dispositivo y verificar el correcto funcionamiento y la conexión de los sensores.
            \item Localizar mediante palpación el pulso arterial en los sitios de medición.
            \item Colocar el sensor proximal (1, color rojo) sobre la arteria carótida derecha, en la región lateral del cuello, sujetándolo con la banda elástica, evitando ejercer presión excesiva.
            \item Colocar el sensor distal (2, color rosa) sobre la arteria radial derecha, en la cara interna de la muñeca, sujetandolo con la banda elástica de forma análoga.
            \item Verificar que ambos sensores esten colocados adecuadamente. El dispositivo indicará si alguno de los sensores no se encuentra haciendo buen contacto.
        \end{itemize}

        \item \textbf{Adquisición de datos:}
        \begin{itemize}
            \item Completar los datos del paciente requeridos.
            \item Presionar el botón \textbf{“Iniciar medición”} para comenzar la adquisición.
            \item Una vez obtenido el resultado, presionar el botón \textbf{“Guardar Medición”} para almacenarlo en el sistema para su posterior análisis.
            \item En caso de observar la señal pletismográfica inestable o con artefactos por movimiento, reacomodar los sensores y repetir la medición.
        \end{itemize}

        \end{enumerate}
    



        \subsubsection{Análisis de repetitibilidad}
        Para determinar la precisión y la variabilidad intra-sujeto de las mediciones, se realizó un análisis de repetibilidad. El estudio se llevó a cabo sobre un único sujeto, siguiendo estrictamente el protocolo de medición previamente establecido.

        El sujeto evaluado fue un voluntario sano, sin antecedentes de enfermedad cardiovascular ni factores de riesgo conocidos. Durante el procedimiento se mantuvieron constantes las condiciones ambientales, la postura corporal y el estado de reposo previo a cada adquisición, con el fin de minimizar posibles variaciones fisiológicas que pudieran influir en los resultados. Todas las mediciones se realizaron en una misma sesión, evitando cambios significativos en el estado hemodinámico.

        Se analizaron dos escenarios experimentales. En primer lugar, se realizaron 10 mediciones consecutivas sin retirar ni modificar la posición de los sensores entre adquisiciones. Posteriormente, se efectuaron otras 10 mediciones consecutivas, retirando completamente y recolocando los sensores antes de cada adquisición. La comparación entre ambas condiciones permitió discriminar qué proporción de la variabilidad total se atribuye al reposicionamiento de los sensores y a la intervención del operador, y cuál corresponde a la variabilidad intrínseca del sistema electrónico y del algoritmo de procesamiento.

        Para cada conjunto de mediciones se calculó la media, el desvío estándar y el coeficiente de variación (CV\%), definido como:

        \[
        CV(\%) = \frac{\sigma}{\mu} \times 100
        \]

        \vspace{0.2cm}


        El coeficiente de variación se utilizó como métrica principal de repetitibilidad, ya que permite normalizar la dispersión respecto del valor medido y facilita la comparación entre ambas condiciones experimentales.




        \subsubsection{Evaluación frente a rangos fisiológicos} 
        Con el objetivo de evaluar la coherencia fisiológica de los valores obtenidos con el dispositivo desarrollado, se realizó una comparación de los mismos con los rangos de PWV reportados en la literatura \cite{hossain2022pwv}. Para ello, se seleccionaron 20 sujetos sanos distribuidos en dos grupos etarios diferenciados. El primer grupo estuvo conformado por 10 participantes con edades comprendidas entre 20 y 25 años, mientras que el segundo grupo incluyó 10 participantes entre 55 y 60 años. Ninguno de los sujetos presentaba antecedentes de enfermedad cardiovascular ni factores de riesgo conocidos. Al igual que en el análisis de repetitibilidad, todas las mediciones se realizaron siguiendo estrictamente el protocolo previamente descripto.

        Para cada grupo etario se calculó la media ($\bar{x}$) y el desvío estándar ($s$) de las mediciones obtenidas. Para evaluar si la media muestral difería significativamente del valor de referencia reportado en la literatura ($\mu_0$), se aplicó una prueba t de Student para una muestra (One-Sample t-test). La hipótesis nula ($H_0$) establece que no existen diferencias significativas entre la media medida y el valor de referencia, mientras que la hipótesis alternativa ($H_1$) plantea que si. El estadístico de prueba se calculó como:

        \[t = \frac{\bar{x} - \mu_0}{\frac{s}{\sqrt{n}}}\]

        \vspace{0.4cm}

        donde $\bar{x}$ es la media muestral, $\mu_0$ el valor de referencia, $s$ el desvío estándar de la muestra y $n$ el tamaño muestral. El valor de $t$ obtenido se comparó con la distribución t de Student con $n-1$ grados de libertad.

        \vspace{0.2cm}

        El análisis se realizó considerando un nivel de significancia $\alpha = 0{,}05$. Un valor de $p > 0{,}05$ se interpreta como ausencia de diferencias estadísticamente significativas respecto del valor de referencia, lo que sugiere coherencia entre las mediciones del dispositivo y los rangos fisiológicos reportados.  Por el contrario, un valor de $p < 0{,}05$ indicaría una discrepancia estadísticamente significativa.





        \subsubsection{Comparación frente a equipo gold standard}
        La validación instrumental del dispositivo Stiffio se realizó en el marco de una colaboración con el Instituto Cardiovascular de Buenos Aires (ICBA), institución de referencia en investigación clínica y atención cardiovascular. El ICBA facilitó el uso del equipo \textit{Arteriograph}, reconocido internacionalmente como método de referencia para la estimación no invasiva de la velocidad de onda de pulso.

        El \textit{Arteriograph} estima la PWV aórtica mediante un método oscilométrico basado en la medición de variaciones de presión obtenidas a través de un manguito braquial. A partir del análisis de la onda de presión y sus reflexiones, el sistema calcula la velocidad de propagación de la onda a lo largo de la aorta. Dado que este procedimiento se basa en señales de presión arterial central, mientras que Stiffio determina la PWV a partir de señales pletismográficas registradas en dos sitios arteriales periféricos, ambos dispositivos no evalúan exactamente el mismo trayecto arterial ni emplean el mismo principio físico de medición. En consecuencia, no se espera una coincidencia absoluta entre los valores obtenidos, sino una concordancia fisiológica y una correspondencia dentro de rangos clínicamente aceptables. En este sentido, se espera que sujetos clasificados dentro del rango fisiológico normal por el \textit{Arteriograph} presenten, de manera consistente, valores compatibles con el rango considerado normal según los criterios establecidos para Stiffio, aun cuando puedan existir diferencias numéricas atribuibles a la metodología empleada por cada sistema.

        La evaluación se llevó acabo con tres sujetos, realizando tres mediciones consecutivas por cada uno con ambos dispositivos. Los participantes fueron adultos jóvenes de entre 20 y 25 años de edad, sin antecedentes cardiovasculares conocidos ni diagnóstico de patología arterial. Todas las mediciones se realizaron siguiendo el protocolo de medicion del respectivo equipo.
 

        Por otro lado, si bien el \textit{Arteriograph} proporciona un valor de HR como parámetro adicional durante la medición, dicho valor se obtiene como resultado derivado del registro oscilométrico y no mediante monitoreo continuo latido a latido. Dado que Stiffio estima la frecuencia cardíaca en tiempo real a partir de la señal pletismográfica, se incorporó un oxímetro de pulso comercial (\textit{Coronet POD-2}) como referencia adicional para la validación de HR. Este dispositivo, al basarse también en fotopletismografía y ofrecer medición continua en tiempo real, resulta metodológicamente más comparable al principio de funcionamiento de Stiffio para la evaluación de este parámetro.




% RESULTADOS =======================================================================================================
\section{Resultados}

\subsection{Evaluación del PWV}

Los resultados correspondientes al análisis de repetibilidad se presentan en las Tablas~\ref{tab:rep_sin} y~\ref{tab:rep_con}, donde se muestran las mediciones obtenidas bajo las dos condiciones experimentales: sin reposicionamiento y con reposicionamiento de sensores, respectivamente.

\begin{table}[H]
\centering
\caption{Mediciones de PWV sin reposicionamiento de sensores.}
\label{tab:rep_sin}
\begin{tabular}{c c}
\hline
Medición & PWV (m/s) \\
\hline
1  & 8.0 \\
2  & 8.3 \\
3  & 8.4 \\
4  & 8.6 \\
5  & 9.2 \\
6  & 8.0 \\
7  & 8.4 \\
8  & 9.0 \\
9  & 8.4 \\
10 & 8.1 \\
\hline
Media & 8.44 \\
Desvío estándar & 0.40 \\
CV (\%) & 4.7\% \\
\hline
\end{tabular}
\end{table}


\begin{table}[H]
\centering
\caption{Mediciones de PWV con reposicionamiento de sensores.}
\label{tab:rep_con}
\begin{tabular}{c c}
\hline
Medición & PWV (m/s) \\
\hline
1  & 9.1 \\
2  & 8.0 \\
3  & 8.0 \\
4  & 7.6 \\
5  & 8.3 \\
6  & 9.3 \\
7  & 8.0 \\
8  & 7.8 \\
9  & 7.7 \\
10 & 7.6 \\
\hline
Media & 8.14 \\
Desvío estándar & 0.64 \\
CV (\%) & 7.9\% \\
\hline
\end{tabular}
\end{table}

Bajo la condición sin reposicionamiento, el coeficiente de variación obtenido fue de 4.7\%, lo que indica una buena repetibilidad del sistema cuando la posición de los sensores se mantiene constante. En la condición con reposicionamiento, el coeficiente de variación fue de 7.9\%, evidenciando un incremento en la variabilidad asociado a la recolocación manual de los sensores. No obstante, el valor permanece por debajo del 10\%, rango comúnmente considerado aceptable en mediciones fisiológicas, lo que sugiere que el sistema mantiene un desempeño estable aun ante cambios en la colocación.


\vspace{0.4cm}

Los valores individuales de PWV obtenidos en las mediciones realizadas a los diferentes sujetos se presentan en la Tabla \ref{tab:mediciones}, divididos según su grupo etareo.

\begin{table}[H]
\centering
\caption{Valores individuales de PWV obtenidos en la medición de diferentes sujetos.}
\label{tab:mediciones}
\begin{tabular}{c c c}
\hline
Grupo & Sujeto & PWV (m/s) \\
\hline
\multirow{10}{*}{20--25 años} 
& S1  & 8,3 \\
& S2  & 8,0 \\
& S3  & 7,4 \\
& S4  & 7,2 \\
& S5  & 7,3 \\
& S6  & 7,4 \\
& S7  & 6,9 \\
& S8  & 9,6 \\
& S9  & 6,7 \\
& S10 & 10,0 \\
\hline
\multirow{10}{*}{55--60 años} 
& S11 & 6,7 \\
& S12 & 8,1 \\
& S13 & 9,4 \\
& S14 & 6,5 \\
& S15 & 8,3 \\
& S16 & 10,5 \\
& S17 & 9,2 \\
& S18 & 7,1 \\
& S19 & 8,4 \\
& S20 & 7,3 \\
\hline
\end{tabular}
\end{table}


Para el grupo joven, la media obtenida fue $7{,}88 \pm 1{,}12$ m/s, mientras que el valor de referencia reportado por Liu et al. fue $7{,}76 \pm 1{,}75$ m/s. El resultado del t-test fue $p = 0{,}74$, indicando que no existen diferencias estadísticamente significativas entre ambas medias.

Para el grupo adulto, la media obtenida fue $8{,}15 \pm 1{,}29$ m/s, mientras que el valor de referencia fue $8{,}82 \pm 1{,}51$ m/s. El resultado del t-test fue $p = 0{,}13$, indicando que no existen diferencias estadísticamente significativas entre ambas medias.

En ambos casos, las medias se encuentran dentro del rango de referencia reportado en la literatura, lo que respalda la validez fisiológica de las mediciones obtenidas con el dispositivo.


\vspace{0.4cm}


En cuanto a la comparación entre el dispositivo desarrollado y el equipo de referencia \textit{Arteriograph}, los resultados obtenidos con ambos métodos se presentan en la Tabla~\ref{tab:validacion_pwv_arteriograph}.


\begin{table}[H]
\centering
\caption{Comparación de mediciones de PWV entre el dispositivo desarrollado y el equipo \textit{Arteriograph}.}
\label{tab:validacion_pwv_arteriograph}
\begin{tabular}{cccccc}
\hline
Sujeto & Medición & PWV (ref) [m/s] & PWV (disp) [m/s] & Diferencia [m/s] \\
\hline
1 & 1 & 5.50 & 8.30 & 2.80 \\
1 & 2 & 6.30 & 8.00 & 1.70 \\
1 & 3 & 6.30 & 7.90 & 1.6 \\
2 & 1 & 4.50 & 7.40 & 2.90 \\
2 & 2 & 4.50 & 7.20 & 2.70 \\
2 & 3 & 4.70 & 7.20 & 2.50 \\
3 & 1 & 6.10 & 7.30 & 1.20 \\
3 & 2 & 6.00 & 7.40 & 1.40 \\
3 & 3 & 5.90 & 7.30 & 1.40 \\
\hline
\end{tabular}
\end{table}

\vspace{0.1cm}

El análisis comparativo muestra que el dispositivo desarrollado presentó una sobreestimación sistemática de la velocidad de onda de pulso respecto al equipo Arteriograph. La diferencia absoluta entre ambos métodos osciló entre 1.2 y 2.9 m/s, con una diferencia media global de aproximadamente 2.0 m/s.


\vspace{0.4cm}


\subsection{Evaluación del HR}
Los resultados obtenidos en la medición de la frecuencia cardíaca, en comparación con el oxímetro comercial utilizado como referencia, se presentan en la Tabla~\ref{tab:validacion_hr_oximetro}.

Tabla:
\begin{table}[H]
\centering
\caption{Comparación de mediciones de HR entre el dispositivo desarrollado y el oxímetro de pulso}
\label{tab:validacion_hr_oximetro}
\begin{tabular}{cccccc}
\hline
Sujeto & HR (ref) [bpm] & HR (disp) [bpm] & Diferencia [bpm] \\
\hline
1 & 83 & 83 & 0 \\
2 & 78 & 77 & -1 \\
3 & 82 & 83 & 1 \\
4 & 50 & 50 & 0 \\
5 & 44 & 46 & 2 \\
6 & 48 & 52 & 4 \\
7 & 71 & 68 & -3 \\
8 & 71 & 70 & -1 \\
9 & 68 & 68 & 0 \\
10 & 60 & 62 & 2 \\
\hline
\end{tabular}
\end{table}

\vspace{0.1cm}

En todos los casos, la diferencia absoluta fue menor o igual a 3 latidos por minuto, cumpliendo con el criterio de validación adoptado para este parámetro. Los resultados muestran una excelente concordancia entre el dispositivo y el oxímetro de pulso, con diferencias mínimas que se encuentran dentro del margen aceptado para este tipo de medición.






% DISCUSIÓN ========================================================================================================
\section{Discusión}

\subsection{Análisis de resultados}

Los resultados obtenidos permiten analizar el desempeño del dispositivo desde tres dimensiones principales: repetibilidad, coherencia fisiológica y concordancia con métodos de referencia.

En términos de repetibilidad, los coeficientes de variación obtenidos (4.7\% sin reposicionamiento y 7.9\% con reposicionamiento) indican una estabilidad adecuada del sistema. El incremento de la variabilidad al recolocar los sensores sugiere que la estimación del tiempo de tránsito es sensible a pequeñas modificaciones geométricas en la adquisición de señal, aunque dicha variabilidad se mantiene dentro de rangos aceptables para mediciones fisiológicas no invasivas. Este comportamiento es consistente con sistemas basados en señales pletismográficas, donde la morfología de la onda puede verse afectada por la posición del sensor y por las condiciones locales de circulación sanguínea.

En cuanto a la validez fisiológica, las medias obtenidas para ambos grupos etarios se ubicaron dentro de los rangos reportados en la literatura y no mostraron diferencias estadísticamente significativas respecto a los valores de referencia. Este hallazgo sugiere que el dispositivo es capaz de capturar tendencias relacionadas con la rigidez arterial asociada a la edad, manteniendo coherencia con la fisiología cardiovascular conocida.

Por otra parte, la comparación con el equipo Arteriograph evidenció una sobreestimación sistemática de la PWV por parte del dispositivo desarrollado. Esta diferencia puede explicarse por las variaciones metodológicas entre ambos sistemas. Mientras que el Arteriograph estima la PWV aórtica central mediante análisis oscilométrico de la onda de presión y sus reflexiones, Stiffio calcula la PWV periférica a partir de señales pletismográficas registradas en dos sitios arteriales superficiales. En consecuencia, no se evalúa el mismo trayecto arterial ni se emplea el mismo principio físico de medición, lo cual puede traducirse en valores sistemáticamente distintos. No obstante, más allá del sesgo observado, se mantuvo una relación entre ambas mediciones. Los sujetos con mayor PWV según el método de referencia también presentaron valores superiores en el dispositivo evaluado, lo que indica que el sistema preserva la jerarquía relativa entre sujetos.

En relación con la frecuencia cardíaca, la comparación con el oxímetro de pulso mostró diferencias mínimas y distribuidas de manera simétrica alrededor de cero. La magnitud reducida del error absoluto indica que el sistema logra identificar adecuadamente los picos sistólicos en la señal pletismográfica y estimar con precisión el intervalo entre latidos. Este resultado respalda la correcta detección temporal de eventos cardíacos, aspecto fundamental para el cálculo fiable del tiempo de tránsito y, por ende, de la velocidad de onda de pulso.

En conjunto, los resultados muestran que el dispositivo presenta buena consistencia interna y coherencia fisiológica, aunque con un sesgo sistemático respecto a la técnica de referencia central. Desde el punto de vista funcional, el sistema demuestra capacidad para estimar diferencias temporales entre señales arteriales y traducirlas en valores reproducibles de velocidad de onda de pulso, lo que constituye una base sólida para futuras etapas de optimización y calibración. 



\subsection{Limitaciones y posibles mejoras}
La evaluación preliminar del dispositivo Stiffio mostró que, en términos generales, las mediciones evidenciaron coherencia fisiológica y consistencia intra-sujeto. Esto indica que el dispositivo es capaz de detectar diferencias temporales entre señales arteriales y traducirlas en estimaciones de velocidad de onda de pulso dentro de los rangos esperables para la población evaluada. No obstante, se identificaron diversos aspectos técnicos y metodológicos que influyen en la calidad de las mediciones obtenidas.

Por un lado, el dispositivo presenta sensibilidad al ruido por movimiento. Artefactos generados por respiraciones profundas, deglución o contracciones musculares pueden alterar la estabilidad de la señal y comprometer la detección precisa de los puntos característicos utilizados para el cálculo de la velocidad de onda de pulso. Esto pone de manifiesto la necesidad de optimizar los métodos de filtrado y los algoritmos de detección de eventos no fisiológicos, minimizando la probabilidad de que estos afecten el resultado final.

Por otro lado, si bien el análisis de variabilidad no mostró diferencias marcadas entre mediciones con y sin reposicionamiento de los sensores, existe una limitación inherente asociada a la naturaleza de los sensores y la colocación manual. Dado que el sistema se basa en la adquisición de señales pletismográficas en sitios arteriales específicos, pequeñas variaciones en la posición, en la presión ejercida o en la alineación respecto al eje arterial pueden modificar la morfología de la señal y afectar la determinación del tiempo de tránsito. Esta característica introduce una componente operador-dependiente que debe considerarse tanto en el entrenamiento del usuario como en el diseño del dispositivo. Como posible línea de mejora, podría desarrollarse un sistema de posicionamiento asistido o un algoritmo automático de evaluación de calidad de señal que proporcione retroalimentación en tiempo real.


Otra limitación relevante está asociada al hardware utilizado. Los sensores empleados fueron originalmente diseñados para mediciones en dedo y corresponden a dispositivos de uso principalmente no profesional, lo que puede restringir su desempeño en aplicaciones clínicas más exigentes. Además, la placa ESP utilizada presenta capacidad de procesamiento limitada, lo que impidió implementar técnicas de procesamiento de señal más complejas o computacionalmente demandantes.

Otro factor relevante se relaciona con la estimación de la distancia recorrida por la onda entre los sitios de medición. En este trabajo, dicha distancia fue medida superficialmente utilizando referencias anatómicas externas, lo cual no necesariamente refleja con exactitud la longitud real del trayecto arterial. Además, existen variaciones anatómicas individuales que no se tienen en cuenta, ya que el dispositivo generaliza para todas las personas de la misma altura. Estas variaciones pueden introducir error en el cálculo, constituyendo una limitación metodológica a considerar en futuras iteraciones del dispositivo.


\subsection{Comparación con dispositivos comerciales}
A diferencia de otros dispositivos disponibles en el mercado, Stiffio presenta como principal ventaja su simplicidad instrumental y bajo costo, al basarse en sensores ópticos y no requerir equipamiento voluminoso ni técnicas de alta complejidad operativa. Esta elección tecnológica implica ciertos compromisos en términos de precisión y robustez, pero al mismo tiempo permite una implementación accesible y potencialmente escalable. 

Además, el dispositivo no requiere oclusión arterial transitoria ni compresión elevada, lo que reduce la posible incomodidad para el usuario. Estas características podrían favorecer su utilización en entornos ambulatorios o en esquemas de monitoreo periódico, donde la simplicidad operativa constituye un factor clave.



% CONCLUSIONES =====================================================================================================
\section{Conclusiones}

El presente trabajo demostró la factibilidad técnica de estimar la velocidad de onda de pulso mediante un sistema basado en la medición del tiempo de tránsito entre señales pletismográficas. El dispositivo desarrollado logró obtener mediciones fisiológicamente coherentes en condiciones controladas, validando el enfoque conceptual adoptado. Si bien se identificaron limitaciones metodológicas y técnicas, los resultados obtenidos confirman la viabilidad del sistema.

Para consolidar su aplicabilidad clínica, es necesario ampliar el tamaño muestral, incorporar poblaciones con distintos perfiles hemodinámicos y realizar estudios comparativos con métodos de referencia. No obstante, el desarrollo alcanzado establece una base sólida para la optimización futura del dispositivo y su proyección como alternativa accesible para la evaluación la evaluación de la rigidez arterial. 



% BIBLIOGRAFÍA =====================================================================================================

\newpage
\newpage
\section{Bibliografía}
\renewcommand{\refname}{} % elimina el título automático
\bibliographystyle{IEEEtran}
\bibliography{Referencias}


% ANEXOS ===========================================================================================================
\newpage
\appendix
\etocdepthtag.toc{anexo}

% TÍTULO
\section*{Anexos}
% ANEXO A =================================================================================================================
\section{Asignación de pines del circuito electrónico}
\label{anexo:ConexionPines}

\begin{table}[H]
    \centering
    \caption{Pines de conexión de la Pantalla TFT a la ESP32}
    \label{tab:ConexionPantalla}
    
    % {|l|l|} significa: Línea | Columna Izq | Línea | Columna Izq | Línea
    \begin{tabular}{|l|l|} 
        \hline
        \textbf{Pin Pantalla TFT} & \textbf{Pin ESP32} \\
        \hline
        VCC             & 3V3 \\
        \hline
        GND             & GND \\
        \hline
        CS              & GPIO 5 \\
        \hline
        RESET           & GPIO 4 \\
        \hline
        DC / RS         & GPIO 2 \\
        \hline
        SDI (MOSI)      & GPIO 23 \\
        \hline
        SCK             & GPIO 18 \\
        \hline
        LED             & 3V3 \\
        \hline
        SDO (MISO)      & GPIO 19 \\
        \hline
        T\_CLOCK (Touch)   & GPIO 18 \\
        \hline
        T\_CS (Touch)   & GPIO 15 \\
        \hline
        T\_DIN (Touch)   & GPIO 23 \\
        \hline
        T\_DO (Touch)   & GPIO 19 \\
        \hline
        T\_IRQ (Touch)   & GPIO 27 \\ 
        \hline

    \end{tabular}
\end{table}




\begin{table}[H]
    \centering
    \caption{Pines de conexión de los sensores MAX3012 a la ESP32}
    \label{tab:ConexionSensores}

    \begin{tabular}{|l|l|} 
        \hline
        \textbf{Pin Pantalla TFT} & \textbf{Pin ESP32} \\
        \hline
        VCC             & 3V3 \\
        \hline
        GND             & GND \\
        \hline
        SDA1              & GPIO 21 \\
        \hline
        SCL1           & GPIO 22 \\
        \hline
        SDA2         & GPIO 25 \\
        \hline
        SCL2      & GPIO 26 \\
        \hline

    \end{tabular}
\end{table}

% ANEXO B =================================================================================================================
\newpage
\section{Modelo 3D  de la carcasa}
\label{anexo:ModeloCarcasa}

\begin{figure}[H]
\centering
\includegraphics[width=0.6\textwidth]{images/Carcasa Solid 1.png}
\captionsetup{margin=1.5cm, justification=centering}
\caption{Vista superior del modelo 3D de la carcasa.}
\label{fig:Carcasa Solid 1}
\end{figure}


\begin{figure}[H]
\centering
\includegraphics[width=0.6\textwidth]{images/Carcasa Solid 2.png}
\captionsetup{margin=1.5cm, justification=centering}
\caption{Vista inferior del modelo 3D de la carcasa.}
\label{fig:Carcasa Solid 2}
\end{figure}


\begin{figure}[H]
\centering
\includegraphics[width=0.6\textwidth]{images/Carcasa Solid 3.png}
\captionsetup{margin=1.5cm, justification=centering}
\caption{Vista interior del modelo 3D de la carcasa.}
\label{fig:Carcasa Solid 3}
\end{figure}


% ANEXO C =================================================================================================================
\vspace{1cm}
\section{Modelo 3D del encapsulado de sensores}
\label{anexo:ModeloSensor}

\begin{figure}[H]
\centering
    
    \begin{minipage}[b]{0.45\textwidth}
        \centering
        \includegraphics[width=\textwidth]{images/Sensor Solid.png}
    \end{minipage}
    \hspace{0.01cm} 
    \begin{minipage}[b]{0.45\textwidth}
        \centering
        \includegraphics[width=\textwidth]{images/Sensor Solid 2.png}
    \end{minipage}
    
\captionsetup{margin=1.5cm, justification=centering}
        \caption{Vista frontal (izquierda) y posterior (derecha) del modelo 3D de la pieza frontal para el encapsulado del sensor.}
        \label{fig:Sensor Solid}
        \end{figure}

% ANEXO D =================================================================================================================
\newpage
\section{factor de corrección altura}
\begin{table}[!ht]
    \centering
    \begin{tabular}{|l|l|l|l|l|}
    \hline
        Edad & Sexo & Altura (cm) & Distancia medida (cm) & Factor de corrección \\ \hline
        61 & Femenino & 155 & 65 & 0,4193548387 \\ \hline
        62 & Masculino & 179 & 86 & 0,4804469274 \\ \hline
        22 & Femenino & 168 & 72 & 0,4285714286 \\ \hline
        22 & Femenino & 175 & 73 & 0,4171428571 \\ \hline
        22 & Femenino & 155 & 69 & 0,4451612903 \\ \hline
        22 & Femenino & 159 & 72 & 0,4528301887 \\ \hline
        22 & Femenino & 180 & 80 & 0,4444444444 \\ \hline
        23 & Masculino & 180 & 81 & 0,45 \\ \hline
        23 & Masculino & 174 & 74 & 0,4252873563 \\ \hline
        22 & Masculino & 177 & 81 & 0,4576271186 \\ \hline
        23 & Masculino & 179 & 78 & 0,4357541899 \\ \hline
        22 & Masculino & 185 & 80 & 0,4324324324 \\ \hline
        23 & Masculino & 177 & 75 & 0,4237288136 \\ \hline
        23 & Masculino & 192 & 80 & 0,4166666667 \\ \hline
        24 & Femenino & 154 & 64 & 0,4155844156 \\ \hline
        24 & Femenino & 170 & 73 & 0,4294117647 \\ \hline
        25 & Femenino & 159 & 71 & 0,4465408805 \\ \hline
        24 & Femenino & 174 & 82 & 0,4712643678 \\ \hline
        22 & Femenino & 158 & 72 & 0,4556962025 \\ \hline
        22 & Femenino & 164 & 70 & 0,4268292683 \\ \hline
        23 & Masculino & 167 & 70 & 0,4191616766 \\ \hline
        22 & Femenino & 159 & 66 & 0,4150943396 \\ \hline
        24 & Femenino & 162 & 71 & 0,4382716049 \\ \hline
        23 & Masculino & 178 & 78 & 0,4382022472 \\ \hline
        23 & Femenino & 168 & 74 & 0,4404761905 \\ \hline
        23 & Femenino & 165 & 68 & 0,4121212121 \\ \hline
        25 & Masculino & 176 & 81 & 0,4602272727 \\ \hline
        62 & Masculino & 187 & 78 & 0,4171122995 \\ \hline
        61 & Femenino & 157 & 67 & 0,4267515924 \\ \hline
        60 & Masculino & 180 & 82 & 0,4555555556 \\ \hline
        60 & Femenino & 150 & 62 & 0,4133333333 \\ \hline
        20 & Femenino & 161 & 72 & 0,4472049689 \\ \hline
        ~ & ~ & ~ & ~ & ~ \\ \hline
        ~ & ~ & ~ & Factor de correccion final & 0,4361964921 \\ \hline
    \end{tabular}
\end{table}


% ANEXO E =================================================================================================================
\newpage
\section{Reporte PDF}
\label{anexo:ReportePDF}


\begin{figure}[H]
\centering
\includegraphics[width=0.9\textwidth]{images/Reporte PDF.png}
\captionsetup{margin=1.5cm, justification=centering}
\caption{Modelo de reporte de medición exportable.}
\label{fig:ReportePDF}
\end{figure}

\end{document}